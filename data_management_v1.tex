\documentclass[10pt]{NSF}

\begin{document}

\section{Data Management Plan}
\label{sec:data_plan}

\subsection{Types of Data}

This project will generate the following types of data:
\begin{itemize}
\item{Engineering drawings and circuit diagrams for custom printed
  circuit boards.}
\item{Field programmable gate array (FPGA) firmware blocks and
  compiled bitstreams.}
\item{Monitor and control (M\&C) software and implementations of radio
  frequency interference (RFI) excision algorithms.}
\item{Laboratory- and telescope-generated high-level test data
  (e.g. digitized raw voltages and detected spectra).}
\item{A database of common sources of RFI and their characteristics.}
\item{Technical documents, memos, and publications.}
\item{Educational curricular materials.}
\end{itemize}

\subsection{Data Formats and Standards}

We will use common data formats wherever possible to facilitate broad
dissemination and use of our work.  This will include:
\begin{itemize}
\item{\textbf{Anything for engineering data?}}
\item{\textbf{What are the technical terms for the FPGA languages?  I
    think BOF is pretty CASPER (and maybe ROACH2) specific...}}
\item{C++ and Python for M\&C software; Python, C++, CUDA, and
  \textbf{FPGA language} for RFI excision algorithms.}
\item{High-level test data will primarily be in the FITS format.
  SDFITS is an implementation for spectral line data supported by GBO;
  PSRFITS is an implementation for pulsar data supported by numerous
  third-party software packages.  Voltage data will in the GUPPI RAW
  format, which is used for other CASPER-based back-ends and supported
  by numerous third-party software packages.}
\item{SQL will be used for the RFI database.}
\item{Documents, including educational materials, will be written in
  portable formats such as PDF.}
\end{itemize}

\subsection{Usage and Distribution Policies}

A public-facing project webpage will be hosted by GBO for
communicating results of our work to a scientific, technical, and
general audience.  It will be updated with status reports upon the
completion of key milestones, and will include links to
publicly-available data products.  An internal wiki page will be
hosted by GBO for recording meeting minutes, internal project memos,
and preliminary data products.
\begin{itemize}
\item{\textbf{Policy for engineering data?}}
\item{Firmware blocks will be made publicly available through
  inclusion in the CASPER library and released under \textbf{Specific
    license?}.  Complete firmware bitstreams will be \textbf{What's
    the policy here?}}
\item{RFI excision algorithms implemented as firmware blocks will be
  made publicly available through inclusion in the CASPER library and
  released under \textbf{Specific license?}.  Software implementations
  will be made publicly available via GitHub or a similar service and
  released under \textbf{specific license?}}
\item{Test-data are may be many tens to hundreds of gigabytes in size,
  and will be hosted on spinning-disk at GBO.  Data generated during
  initial testing and debugging will be retained for use by the
  project time.  Final validation and verification data will be made
  publicly available.  High-level descriptions and meta-data of final
  validation and verification data will be published on the public
  project webpage.  Final validation and verification data sets will be
  made available via FTP.}
\item{RFI database will be publicly accessible via a GBO-hosted
  webpage.  Community contributions will be welcome with editing
  privileges granted on a case-by-case basis via coordination with the
  PI and the GBO computing division.}
\item{Final commissioning reports, technical publications, and
  curricular material will be made freely available via the public
  project webpage.}
\end{itemize}

\subsection{Data Archiving}

GBO will curate and archive data.  Documentation, firmware blocks and
bitstreams, test data, public databases, and public and internal
webpages will be stored locally on spinning disks that undergo regular
back-up as part of normal GBO computing division activities.  Local
revision control repositories will be used to track changes in
firmware and software.

\end{document}
