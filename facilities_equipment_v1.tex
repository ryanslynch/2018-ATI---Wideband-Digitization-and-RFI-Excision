\documentclass[10pt]{NSF}

\usepackage{mypack}

\begin{document}

\section{Facilities, Equipment, and Other Resources}

\subsection{Green Bank Observatory}
\label{sec:gbo}

The Green Bank Observatory (GBO) is a Federally-Funded Research and
Development Center whose mission is to enable forefront research into
the radio universe, by providing world-leading telescopes,
instrumentation and expertise, training the next generation of
scientists and engineers, and promoting astronomy to foster a more
scientifically literate society.

GBO operates the Robert C.\ Byrd Green Bank Telescope (GBT), the
world's largest fully steerable telescope.  The GBT provides nearly
continuous frequency coverage from $0.3$--$116\; \GHz$ and a diverse
suite of instruments, including two CASPER-based digital spectrometers
with spectral line and pulsar observing modes.  GBO also operates a
number of additional research-class telescopes, including a 20-m
telescope equipped with a ROACH1-based digital back-end that has been
used to prototype real-time radio frequency interference (RFI)
techniques.

\subsubsection{Digital Electronics Group}
\label{sec:digital_group}

GBO digital engineers have a combined *** years of experience
designing, building, testing, deploying, and operating world-leading
instruments for radio astronomy.  This includes expertise in the
design of low-power and low-noise electronics and the Xilinx Vivado
tool flow for complex firmware design.  Our group has also become
active in designing and implementing real-time RFI excision algorithms
in firmware design, and have recently complete the first real-time
CASPER-based system for mitigating impulsive RFI such as airport
RADAR.

\subsubsection{Radio Quiet Zones}
\label{sec:quiet_zone}

GBO is located in the heart of the 13,000 square-mile National Radio
Quiet Zone (NRQZ) and the West Virginia Radio Astronomy Zone (WVRAZ).
The NRQZ was established by the Federal Communications Commission and
the Interdepartment Radio Advisory Committee in 1958 to minimize
possible harmful interference to the National Radio Astronomy
Observatory NRAO in Green Bank, WV (now GBO) and the radio receiving
facilities for the United States Navy in Sugar Grove, WV.  The NRQZ
provides regulatory protection by requiring coordination between GBO
and all new or modified, permanent, fixed, licensed transmitters inside
the NRQZ, as specified for federal transmitters by NTIA manual section
8.3.9 and for non-federal transmitters by the FCC in 47 CFR section
1.924.  The NRQZ office at GBO ensures that the power flux density of
transmitters within the zone does not exceed certain
frequency-dependent thresholds, and works with with applicants to
mitigate any non-compliant devices.  Mitigation methods include the use
of directional antennas, locating transmitters to areas that provide
terrain shielding, or selecting a different frequency where the
transmitter power is within limits.

The WVRAZ provides stricter protection within a more limited
geographic area.  Within two miles of GBO all electrical devices that
interfere with the operation of GBO are prohibited.  This includes the
use of wireless internet, Bluetooth, and broadband transmitters such
as microwave ovens, to name but a few.  Between a two-mile and
ten-mile radius from GBO, strict power dissipation requirements are in
place.

The NRQZ and WVRAZ are a unique and invaluable resource for GBO, but
cannot protect against all forms of interference.  Notable exceptions
are the growing use of car RADAR collision avoidance systems,
aircraft, and communications satellites.  This necessitates observing
systems that produce high-quality data in the presence of RFI, such as
the one we will build as part of this project.

\subsubsection{Anechoic Test Chamber}

A shielded anechoic chamber measuring $15' \times 15' \times 37'$ is
located in the Jansky Laboratory at GBO.  The chamber serves as both a
far-field antenna test range and for evaluating new equipment for
compliance with limits on self-generated RFI.  The GBO Interference
Protection Group evaluates all equipment and devises mitigation
techniques to ensure compliance as needed.  The anechoic chamber will
be used to evaluate all hardware developed as part of this project,
including all equipment that will be installed near the GBT radio
receivers.

\subsubsection{Machine Shop}

GBO operates a full machine shop that can build custom fabricated
parts as needed.  We do not anticipate heavy use of the machine shop
for this project, but could rely on it to build RFI-shielded casing
for the electronics that will be placed in the GBT receiver and
equipment rooms.

\subsubsection{Education and Public Outreach}

GBO has a nationally recognized, fully-developed education and public
outreach (EPO) program.  The Green Bank Science Center hosts over
50,000 visitors each year Over 3,000 students annually take part in
various on-site educational programs.  GBO has a variety of long- and
short-term housing, including a dormitory that sleeps up to 60
students (which will be used for the two-week internship) and on-site
housing for summer research students (which will be used for the
undergraduate student supported by this proposal).  GBO is a member of
the NSF-INCLUDES Alliance First 2 Network.

\subsubsection{Project Management}

Anything to say here?

\subsection{The University of California, Berkeley}

\end{document}
