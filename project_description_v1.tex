\documentclass[10pt]{NSF}

\title{TITLE}

\begin{document}

\section{Project Description}
\label{sec:project_description}

\begin{itemize}
\item{\textbf{Note: 15-page limit}}
\item{\textbf{Note: A separate, 2-page data management plan can
      include details on standards used for data and metadata, and
      policies for accessing, sharing, and re-using data}}
\item{\textbf{Note: A separate resources and facilities section can
      include a ``description of the internal and external resources
      (both physical and personnel)'', and this may be a good place to
      discuss local expertise.  We must also include biographical
      sketches that list education, professional preparation, and
      related ``products'', somewhat akin to a CV}}
\end{itemize}

\subsection{Overview}
\label{sec:overview}

\begin{itemize}
\item{\textbf{Section lead(s):} Ryan Lynch}
\item{\textbf{Target length:} 3/4 page}
\item{Brief description of:
    \begin{itemize}
    \item{Context within the development of the UWB Rx}
    \item{Key ideas/goals of the project}
    \item{Scientific and technical motivation}
    \item{Innovative/transformative impact of the project}
    \item{Benefits to the wider astronomical community}
    \item{Broader impacts, including enhanced infrastructure for
        research and education and impact on STEM education}
    \end{itemize}}
\end{itemize}
  
\subsection{Intellectual Merit}
\label{sec:IM}

\subsubsection{Motivation}
\label{sec:motivation}

\begin{itemize}
\item{\textbf{Section lead(s):} Ryan Lynch, digital group}
\item{\textbf{Target length:} 3 pages}
\item{\textbf{Note: Highlight ``expected significance'' here}}
\item{\textbf{Note: Highlight ``objectives'' here}}
\item{Scientific Motivation (Ryan Lynch and Scott Ransom)}
  \begin{itemize}
  \item{Direct detection and characterization of low-frequency
      gravitational wave Universe (i.e., NANOGrav)}
    \begin{itemize}
    \item{Brief explanation of pulsar emission and ISM effects}
    \item{Expected impact of UWB Rx for GW detection, including
        mention of Michael Lam's works}
    \end{itemize}
  \item{Broadband spectral studies of transients}
    \begin{itemize}
    \item{Brief description of magnetars and FRBs}
    \item{Importance of wide instantaneous bandwidth}
    \end{itemize}
  \item{Additional science drivers (scientific staff; radio
      recombination lines?  astrochemistry?)}
  \end{itemize}
\item{Technical Motivation (Digital group (+ BTL/CASPER?))}
  \begin{itemize}
  \item{\textbf{Randy: please provide your thoughts on what best goes
        under this section}}
  \item{Importance of digitizing at RF}
    \begin{itemize}
    \item{RFI resistance (i.e. high dynamic range, reducing analog
        components)}
      \begin{itemize}
      \item{Talk about limitations of existing IF system and VEGAS here}
      \end{itemize}
    \item{Improved stability, reliability?}
    \end{itemize}
  \item{``Sharing the spectrum'' (i.e. RFI excision)}
  \end{itemize}
\end{itemize}

\subsubsection{Innovation}
\label{sec:innovation}

\begin{itemize}
\item{\textbf{Section lead(s):} Digital group (+ BTL/CASPER?)}
\item{\textbf{Target length:} 4 pages}
\item{\textbf{Note: Highlight ``relationship to present state of
      knowledge'' here}}
\item{\textbf{Note: To the extent possible, explicitly describe work
      to be undertaken and/or major activities (can expand in
      \S\ref{sec:plan} as needed)}}
\item{\textbf{Randy: please provide your thoughts on what best goes
      under this section}}
\item{New hardware?}
\item{Firmware development}
  \begin{itemize}
  \item{Fast sampling}
  \item{Increased bit-depth/dynamic range}
  \item{Dealing with bandpass slope/selecting significant bits?}
  \end{itemize}
\item{Protocols/formats for high data rates}
  \begin{itemize}
  \item{Packetization}
  \item{10 $\rightarrow$ 40 $\rightarrow$ 100 GbE}
  \item{\textbf{Question for digital group: How would we break up band
        (i.e. subband the way GUPPI and VEGAS do in coherent DD modes)
        and transmit to HPCs?}}
  \item{\textbf{Note: We should talk to Chris and computing group
        about network infrastructure}}
  \end{itemize}
\item{Active RFI excision}
  \begin{itemize}
  \item{MAD and SK algorithms}
  \item{Machine learning algorithms}
    \begin{itemize}
    \item{Talk about new chips/architecture here?}
    \end{itemize}
  \end{itemize}
\item{Interference compliance}
  \begin{itemize}
  \item{Design of low-power, non-interfering electronics}
  \item{Shielding (w/ input from Carla?)}
  \end{itemize}
\item{Cooling? (w/ input from mechanical/works divisions?)}
\item{Commensal/multi-backend/multi-mode observing?}
  \begin{itemize}
  \item{\textbf{Note: We should talk with software group about
        software backends}}
  \end{itemize}
\item{\textbf{Note: We could include impact on key science drivers
      here}}
\end{itemize}

\subsection{Broader Impacts}
\label{sec:BI}

\textbf{Note: We should decide if we want to support a postdoc,
  intern, or grad student.  This would most likely be in engineering.
  If so, we can place this under sections for ``Development of a
  Competitive Workforce''}

\textbf{Note: Ryan will talk to Sue Ann about whether we can naturally
  include any EPO components (within budget)}

\subsubsection{Commitment to the Public}
\label{sec:commitment}

\begin{itemize}
\item{\textbf{Section lead(s):} Ryan Lynch}
\item{\textbf{Target length}: 3/4 page}
\item{Facility-supported, general-user, open-skies instrumentation}
\item{Make all designs, firmware, software, and RFI excision
    algorithms publicly available}
\end{itemize}

\subsubsection{Enhanced Infrastructure for Research and Education}
\label{sec:infrastructure}

\begin{itemize}
\item{\textbf{Section lead(s):} Ryan Lynch + scientific staff, digital
    group}
\item{\textbf{Target length:} 2pages}
\item{Reiterate importance to UWB Rx project}
  \begin{itemize}
  \item{Include impact metrics for NANOGrav, pulsar, transient, and
      other science areas}
  \end{itemize}
\item{``Pilot program'' for GBT IF system upgrades}
  \begin{itemize}
  \item{Enable instantaneous use of full bandwidth for all existing
      (single-feed?) receivers}
    \begin{itemize}
    \item{Focus on the science this would enable (e.g. astrochemistry,
        extragalactic surveys)}
    \end{itemize}
  \item{Provide maximum flexibility when balancing bandwidth vs number
      of pixels for camera program}
    \begin{itemize}
    \item{Mention Argus+ and any other camera programs?}
    \end{itemize}
  \item{Incorporate active RFI excision at all frequencies}
    \begin{itemize}
    \item{Mention car radar and any other new, high-frequency sources
        of RFI (w/ input from IPG)}
    \end{itemize}
  \item{Provide greater resistance to RFI through increased bit depth
      and by minimizing analog components}
  \item{Improve reliability and reduce operational complexity by
      minimizing components in signal path}
  \item{Update IF system with state-of-the-art technology}
  \item{Create a model for fast, modular upgrades as new technology
      emerges}
  \end{itemize}
\item{Relevance for other instruments}
  \begin{itemize}
  \item{Highlight ngVLA, SKA, and emphasize that all products of
      research will be shared freely}
    \begin{itemize}
    \item{See if Jay and/or BTL/CASPER know of any existing plans for
        RF digitization at these or other observatories}
    \end{itemize}
  \end{itemize}
\end{itemize}

\subsection{Project Plan and Timeline}
\label{sec:plan}

\begin{itemize}
\item{\textbf{Section lead(s):} Laura Jensen}
\item{\textbf{Target length:} 2 pages}
\end{itemize}

\subsubsection{Work to be Undertaken OR Key Milestones and Evaluation}
\label{sec:milestones}

\begin{itemize}
\item{Should align with activities identified in
    \S\ref{sec:innovation}}
\item{Should also include metrics for success and a plan for evaluation}
\end{itemize}


\subsubsection{Proposed Timeline}
\label{sec:timeline}

\begin{itemize}
\item{A graphical timeline, organized by year and work type}
\item{A narrative description of the timeline}
\end{itemize}

\subsection{Results from Prior NSF Support}
\label{sec:prior_support}

\begin{itemize}
\item{\textbf{Section lead(s):} All (as needed)}
\item{\textbf{Target length:} ? (must be $< 5$ pages)}
\item{\textbf{Note: Only needed for PIs or co-PIs with a current NSF
      award or one with an end date in the past five years.}}
\item{For each award:}
  \begin{itemize}
  \item{NSF Award number, amount, and period of support}
  \item{Title of project}
  \item{Summary of completed/proposed work}
    \begin{itemize}
    \item{Intellectual Merit}
    \item{Broader Impacts}
    \end{itemize}
  \item{List of publications}
  \item{Evidence of research projects and their availability}
  \item{Relation of completed work to proposed work (for renewals
      only)}
  \end{itemize}
\end{itemize}

\end{document}
