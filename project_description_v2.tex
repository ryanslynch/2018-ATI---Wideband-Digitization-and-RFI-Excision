\documentclass[10pt]{NSF}

\usepackage{amssymb}
\usepackage{mypack}

\begin{document}

\title{An Ultrawideband Digital Signal Processing System for the Green
  Bank Telescope}
\maketitle

\section{Project Description}
\label{sec:project_description}

\begin{itemize}
\item{\textbf{Note: 15-page limit}}
\item{\textbf{Note: A separate, 2-page data management plan can
      include details on standards used for data and metadata, and
      policies for accessing, sharing, and re-using data}}
\item{\textbf{Note: A separate resources and facilities section can
      include a ``description of the internal and external resources
      (both physical and personnel)'', and this may be a good place to
      discuss local expertise.  We must also include biographical
      sketches that list education, professional preparation, and
      related ``products'', somewhat akin to a CV}}
\end{itemize}

\subsection{Overview}
\label{sec:overview}

We propose to use state-of-the-art technologies to build an
ultrawideband digital signal processing (UWB-DSP) system that will be
integrated into a new radio receiver under development for the Robert
C.\ Byrd Green Bank Telescope (GBT) at the Green Bank Observatory
(GBO).  Our UWB-DSP system will consist of fast, **high** \textbf{(is
  this the right word?)} bit-depth analog-to-digital converters (ADCs)
that will directly sample the entire $0.7$--$4\; \GHz$ radio frequency
(RF) bandwidth of the new ultrawideband receiver (UWBR), bypassing the
GBT's usual system of analog mixers and filters that convert the RF
signal to an intermediate frequency (IF).  The UWB-DSP system will
also make use of new system-on-chip (SOC) architectures to enable
advanced, real-time radio frequency interference (RFI) excision using
statistical and new machine learning (ML) algorithms to identify RFI
in the presence of astronomical signals.  The combination of real-time
RFI removal and high dynamic range digitization as close as possible
to the frontend receiver will make this complete UWB system
**significantly** \textbf{(better word choice?)} more resistant to RFI
than is currently possible with existing technology on the GBT.  This
is crucially important given the experience of a similar UWB system
that have been deployed on the Effelsberg Radio Telescope that were
crippled by strong interference.

The primary science motivation for our UWB system is the direct
detection of low-frequency gravitational waves (GWs) via pulsar
timing.  The system will also allow for new, wideband spectral studies
of fast radio bursts (FRBs), magnetars, and other radio transients, as
well as faster surveys of regions rich in molecular lines at these
frequencies (e.g. HII radio recombination lines and complex chemical
species).  The UWBR itself will deliver a combination of wider
instantaneous bandwidth and lower system noise temperature than was
possible with previous generation technology.  The UWB-DSP system will
use cutting edge ADCs, innovative RFI excision techniques not in use
at any other observatories, and pioneering methods for handling very
high data rates using 100-gigabet ethernet (GbE) protocols.  Our
project will thus pair advanced digital and analog technologies for
the world's largest single-dish radio telescope to enable
transformative scientific advances in cutting edge fields of astronomy
and astrophysics.

The UWB system will be deployed on the GBT as a facilty instrument
open to the full astronomical community via the NSF-funded open skies
program.  This project will also serve as a pilot program for upgrades
to the GBT's existing receivers and IF system.  All of the hardware
design, firmware, and software developed through our efforts will be
made publicly available for use at other observatories, and will be
directly relevant for possible future telescopes such as the Next
Generation Very Large Array (ngVLA) and Square Kilometer Array (SKA).
We will also leverage GBO's leadership in the NSF INCLUDES program to
broaden participation in digital engineering and radio astronomy via
an annual summer camp for undergraduate students.  During this camp
students will directly participate in developing our new RFI excision
project by creating training data sets for machine learning
algorithms.  The students will also be exposed to all wide range of
engineering and scientific disciplines that contribute to the success
of the GBO, and will receive interventions that will increase
retention in STEM fields \textbf{(among rural/first generation college
  students?)}.  This will create a pipeline of students for GBO's
successful summer student programs (including our NSF-funded REU
program), alumni of which have already contributed to the RFI excision
project.  The UWB-DSP project will thus have an extremely broad impact
on the wider scientific community and the next generation of STEM
professionals.

\subsection{Intellectual Merit}
\label{sec:IM}

\subsubsection{Motivation}
\label{sec:motivation}

\subsubsection{Scientific Motivation}
\label{sec:science_motivation}

\alanheading{The Low-Frequency GW Universe} The primary science driver
of the UWB system is the direct detection of nanohertz frequency GWs
via pulsar timing, which is the focus of the NSF-supported North
American Nanohertz Observatory for Gravitational Waves Physics
Frontier Center (NANOGrav PFC).  At these GW frequencies the dominant
source class is expected to be supermassive binary black holes (SMBBH)
in the early stages of inspiral at the centers of galaxies; exotic
sources of GWs such as cosmic strings may also emit in the
$\nHz$-regime.  The NANOGrav PFC is on-track to detect a stochastic GW
background within the next 3--5 years, and is already placing
important constraints on the amplitude of this background that informs
models of SMBBH evolution and coupling to the surrounding galactic
environments.  The detection of individual continuous wave sources is
expected to follow in the coming decade, which will enable
multi-messenger studies of SMBBH systems.  The NANOGrav PFC also
places the most stringent existing limits on the energy density of
cosmic strings.  Other science is obtained ``for free'' as part of
this project, such as pulsar mass measurements, tests of general
relativity, and novel constraints on Solar system planetary
ephemerides.

The NANOGrav PFC uses the GBT and the William E. Gordon telescope at
the Arecibo Observatory (AO) to observe a pulsar timing array (PTA) of
millisecond pulsars (MSPs) distributed across the sky.  The extremely
high rotational stability of MSPs allows them to be used as clocks
whose ``ticks'' are pulse times of arrival (TOAs) that can be
accurately measured \emph{and predicted} with accuracies of
$\lesssim 100\; \ns$ over time scales of decades.  The influence of
GWs at the Earth will cause a $10$--$100\; \ns$ deviation in the TOAs
because of the changing path-length between the observer and the
pulsars.  The quadripolar nature of GWs will specifically cause a
quadripolar angular correlation between pulsars distributed across the
sky, which makes it possible to distinguish between GWs and sources of
TOA deviations (e.g. uncorrelated effects on individual pulsars, or
observatory clock errors or Solar System effects that will have a
monopolar and dipolar angular correlation, respectively).

One of the most important steps in obtaining TOAs with the required
accuracy is measuring and correcting for the effects of the ionized
interstellar medium (ISM) on pulsar signals.  One of these effects is
a dispersirve delay given by
\begin{equation}
  \Delta t_{\rm DM} = k_{\rm DM} \, \DM(t,f)\, \left( f_{\rm l}^{-2} -
  f_{\rm h}^{-2} \right) ,
\end{equation}
where $k_{\rm DM}$ is a physical constant, $f$ is the radio freqnuency
and the subscripts denote a lower and higher frequency, and $\DM(t,f)$
is the dispersion measure, i.e. column density of free electrons
between the pulsar and the Earth.  We emphasize that \DM is both time
and frequency dependent (see \textbf{***}), and thus represents a
noise term \emph{that must be measured at each observing epoch with
  fractional precision of $\sim ***$}.

The NANOGrav PFC currently employees a two-receiver strategy at the
GBT to precisely meausure DM, observing from $0.72$--$0.92\; \GHz$ and
$1.1$--$1.9\; \GHz$.  This approach is suboptimal for several reasons.
First, it effectively doubles the observing time needed to obtain a
single TOA.  Second, for operational reasons these observations are
typically scheduled with a separation of a few days, making it
impossible to resolve DM variations on shorter timescales.  \emph{The
  UWB system will double the observational efficiency of
  high-precision pulsar timing programs while providing a factor of
  \textbf{***} improvement in measurements of DM.  When coupled with
  higher pulsar signal-to-noise from the wider instantaneous
  bandwidth, the NANOGrav PFC's sensitivity to GWs will increase at
  twice the rate as without an UWB system}.  This in turn will
effectively double the volume over which the PTA is sensitive to
individual SMBBHs.

\alanheading{Radio Transients} Fast radio bursts are millisecond
duration radio-frequency pulses that originate beyond the Milky Way.
Their phyiscal origin is one of the most pressing mysteries in
astronomy and will be a major area of research in the coming decade.
To-date, only one FRB has been observed to repeat (FRB 121102), a fact
which has enabled the only precisise interferometric localization of
an FRB to a host galaxy, as well as long-duration study of the
changing characteristics of the bursts (e.g. DM, Faraday rotation
measure, and burst morphology).  With telescope like the Australian
SKA Pathfinder and the Canadian HI Intensity Mapping Experiment poised
to discover dozens (if not hundreds) of new FRBs, more repeaters are
sure to follow.

FRB 121102 shows dramatic variation in spectral shape and burst
morphology from pulse to pulse which may be due to intrinsic or
extrinsic effects.  Any theory regarding the nature of FBR 121102 (and
presumably at least some class of FRBs more generally) must explain
these properties, so they serve as a powerful diagnostic tool.  Bursts
are to be band-limited with variable center frequency and bandwidth
and have been detected at frequencies as high as $8\; \GHz$.  Most
observations are limited by the bandwidth of the receiver, however, so
the only way thus far to investigate the behavior of FRB 121102 over
ultrawide bandwidths has been through simultaneous observations with
multiple telescopes.  This is obviously logistically complicated and
suboptimal.  Our new UWB system will enable spectral studies of FRB
121102 and future repeating FRBs, and will give us a better
understanding of the characteristic bandwidth of the bursts and their
low-frequency cutoff (so far FRB 121102 has not been detected below
$1\; \GHz$).  These will have important implications for the theory of
FRBs.

Radio magnetars --- a sub-class of neutron stars whose emission is
powered by the decay of extremely strong magnetic fields --- represent
another broadband and highly variable radio transient.  To-date only
four radio magnetars have been discovered (out of a larger population
of \textbf{***} magnetars that emit X-rays and gamma-rays).  Their
sporadic emission and variable power-law spectral index, polarization
fraction and position angle, and burst morphology stand in stark
contrast to rotation-powered radio pulsars, and the physical processes
giving rise to their radio emission remains a mystery.  A young,
energetic magnetar are one of the leading explanations for the origin
of FRB 121102, and as with this FRBs studies of radio magnetars over
ultrawide bandwidths have been difficult.  The UWB system will thus be
a unique resource for learning more about multiple populations of
radio transients.

\alanheading{Molecular Line Surveys} \textbf{TODO}

\subsubsection{Technical Motivation}
\label{sec:technical_motivation}

The observations over an ultrawide instantaneous bandwidth needed to
realize the above scientific potential come with a number of technical
challenges.  The frontend receiver must achieve a high aperture
effiency and polarization purity while maintaining a low system noise
temperature.  The entire system from the receiver's low noise
amplifiers through the ADCs must also \emph{share the spectrum},
producing scienticically usable data at frequencies where there is
significant, strong RFI.  GBO's location in the National Radio Quiet
Zone and West Virginia Radio Astronomy Zone gives it unique
interference protection, but many source of RFI, such as satellite
transmitters, are unavoidable (for more inforation on these
interference protection zones see Facilities, Equipment, and Other
Resources).

The frontend receiver is being developed as a research project by GBO
\textbf{with additional support provided by the Gordon and Betty Moore
  Foundation via a grant to the NANOGrav PFC}.  The work supported by
this ATI proposal will focus on the DSP system after the front-end
receiver.

\alanheading{RFI Resistance} The GBT currently uses the Versatile
Green Bank Astronomical Spectrometer (VEGAS, developed in part with
support from \textbf{NSF-***}) as its primary digital backend system.
VEGAS uses eight spectrometer banks each consisting of $2 \times 3\;
\mathrm{Gsps}$ 8-bit ADCs (one for each polarization channel) paired
with a high-performance computer (HPC) equipped with an nVidia
\textbf{GTX 1080} graphical processing units (GPUs).  A relatively
straightforward expansion of the HPC system will be sufficient to
process the full bandwidth provided by the UWBR for pulsar
observations, but this approach comes with significant drawbacks.
Most notably, VEGAS makes use of the GBTs IF system before sampling,
which exposes the UWB system to potential saturation of analog
components including and the RF-over-fiber system, two additional
frequency mixers and bandpass filters, and the VEGAS 8-bit ADCs.  In
recent months GBO staff have become aware of total power instabilities
that are present at a variety of observing bands (most notably L-Band,
i.e. $1$--$2\; GHz$) and digital backend systems in addition to VEGAS.
These instabilities correlate with the presence of strong RFI and have
been isolated to somewhere between the RF-over-fiber system and second
frequency mixer.  The UWB system will be exposed to the same RFI
environment as the L-Band receiver and more.  This band includes
cellular communications, digital television, Global Positioning
System, airport radar and aircraft positioning systems, Iridium, and
Sirius XM Satellite Radio.  \emph{We thus have empirical evidence that
  illustrates the need to minimize the analog components in the UWB
  signal path, sampling RF with sufficient dynamic range to maintain
  linearity across $0.7$--$4\; \GHz$}.

\alanheading{RFI Exscision}


\begin{itemize}
\item{\textbf{Section lead(s):} Ryan Lynch, digital group}
\item{\textbf{Target length:} 3 pages}
\item{\textbf{Note: Highlight ``expected significance'' here}}
\item{\textbf{Note: Highlight ``objectives'' here}}
\item{Scientific Motivation (Ryan Lynch and Scott Ransom)}
  \begin{itemize}
  \item{Additional science drivers (scientific staff; radio
    recombination lines?  astrochemistry?)}
  \end{itemize}
\item{Technical Motivation (Digital group (+ BTL/CASPER?))}
  \begin{itemize}
  \item{\textbf{Randy: please provide your thoughts on what best goes
        under this section}}
  \item{Importance of digitizing at RF}
    \begin{itemize}
    \item{RFI resistance (i.e. high dynamic range, reducing analog
        components)}
      \begin{itemize}
      \item{Talk about limitations of existing IF system and VEGAS here}
      \end{itemize}
    \item{Improved stability, reliability?}
    \end{itemize}
  \item{``Sharing the spectrum'' (i.e. RFI excision)}
  \end{itemize}
\end{itemize}

\subsubsection{Innovation}
\label{sec:innovation}

\begin{itemize}
\item{\textbf{Section lead(s):} Digital group (+ BTL/CASPER?)}
\item{\textbf{Target length:} 4 pages}
\item{\textbf{Note: Highlight ``relationship to present state of
      knowledge'' here}}
\item{\textbf{Note: To the extent possible, explicitly describe work
      to be undertaken and/or major activities (can expand in
      \S\ref{sec:plan} as needed)}}
\item{\textbf{Randy: please provide your thoughts on what best goes
      under this section}}
\item{New hardware?}
\item{Firmware development}
  \begin{itemize}
  \item{Fast sampling}
  \item{Increased bit-depth/dynamic range}
  \item{Dealing with bandpass slope/selecting significant bits?}
  \end{itemize}
\item{Protocols/formats for high data rates}
  \begin{itemize}
  \item{Packetization}
  \item{10 $\rightarrow$ 40 $\rightarrow$ 100 GbE}
  \item{\textbf{Question for digital group: How would we break up band
        (i.e. subband the way GUPPI and VEGAS do in coherent DD modes)
        and transmit to HPCs?}}
  \item{\textbf{Note: We should talk to Chris and computing group
        about network infrastructure}}
  \end{itemize}
\item{Active RFI excision}
  \begin{itemize}
  \item{MAD and SK algorithms}
  \item{Machine learning algorithms}
    \begin{itemize}
    \item{Talk about new chips/architecture here?}
    \end{itemize}
  \end{itemize}
\item{Interference compliance}
  \begin{itemize}
  \item{Design of low-power, non-interfering electronics}
  \item{Shielding (w/ input from Carla?)}
  \end{itemize}
\item{Cooling? (w/ input from mechanical/works divisions?)}
\item{Commensal/multi-backend/multi-mode observing?}
  \begin{itemize}
  \item{\textbf{Note: We should talk with software group about
        software backends}}
  \end{itemize}
\item{\textbf{Note: We could include impact on key science drivers
      here}}
\end{itemize}

\subsection{Broader Impacts}
\label{sec:BI}

\textbf{Note: We should decide if we want to support a postdoc,
  intern, or grad student.  This would most likely be in engineering.
  If so, we can place this under sections for ``Development of a
  Competitive Workforce''}

\textbf{Note: Ryan will talk to Sue Ann about whether we can naturally
  include any EPO components (within budget)}

\subsubsection{Commitment to the Public}
\label{sec:commitment}

The UWB-DSP system will be deployed on GBT as a facility-supported,
general purpose instrument available to all GBT users.  A majority of
GBT time is allocated through the NSF-funded open-skies program, and
is thus open to astronomers anywhere in the world.  The other primary
users of the GBT are the Breakthrough Listen project and the NANOGrav
PFC.  The importance of the UWB system to the NANOGrav PFC has been
explained in \S\ref{sec:science_motivation}, and it will also be
valuable to Breakthrough Listen, as it will allow for faster surveys
for extraterrestrial technosignatures.  Both NANOGrav and Breakthrough
Listen have committed to making data publicly available. (\textbf{(How
  much should we go into detail on this?)}

All of the hardware designs, firmware, and software produces through
this ATI will be made publicly available to the wider astronomical and
radio science communities for use at other facilities.  These products
will also be documented and spread through the wider community through
publicatoin in peer-reviewed journals and presentations at meetings
such as AAS and URSI.  

\begin{itemize}
\item{\textbf{Section lead(s):} Ryan Lynch}
\item{\textbf{Target length}: 3/4 page}
\item{Facility-supported, general-user, open-skies instrumentation}
\item{Make all designs, firmware, software, and RFI excision
    algorithms publicly available}
\end{itemize}

\subsubsection{Enhanced Infrastructure for Research and Education}
\label{sec:infrastructure}

\begin{itemize}
\item{\textbf{Section lead(s):} Ryan Lynch + scientific staff, digital
    group}
\item{\textbf{Target length:} 2pages}
\item{Reiterate importance to UWB Rx project}
  \begin{itemize}
  \item{Include impact metrics for NANOGrav, pulsar, transient, and
      other science areas}
  \end{itemize}
\item{``Pilot program'' for GBT IF system upgrades}
  \begin{itemize}
  \item{Enable instantaneous use of full bandwidth for all existing
      (single-feed?) receivers}
    \begin{itemize}
    \item{Focus on the science this would enable (e.g. astrochemistry,
        extragalactic surveys)}
    \end{itemize}
  \item{Provide maximum flexibility when balancing bandwidth vs number
      of pixels for camera program}
    \begin{itemize}
    \item{Mention Argus+ and any other camera programs?}
    \end{itemize}
  \item{Incorporate active RFI excision at all frequencies}
    \begin{itemize}
    \item{Mention car radar and any other new, high-frequency sources
        of RFI (w/ input from IPG)}
    \end{itemize}
  \item{Provide greater resistance to RFI through increased bit depth
      and by minimizing analog components}
  \item{Improve reliability and reduce operational complexity by
      minimizing components in signal path}
  \item{Update IF system with state-of-the-art technology}
  \item{Create a model for fast, modular upgrades as new technology
      emerges}
  \end{itemize}
\item{Relevance for other instruments}
  \begin{itemize}
  \item{Highlight ngVLA, SKA, and emphasize that all products of
      research will be shared freely}
    \begin{itemize}
    \item{See if Jay and/or BTL/CASPER know of any existing plans for
        RF digitization at these or other observatories}
    \end{itemize}
  \end{itemize}
\end{itemize}

\subsubsection{Broadening Participation}
\label{sec:participation}

\begin{itemize}
\item{\textbf{Section lead(s):} Sue Ann Heatherly}
\item{\textbf{Target length:} 2 pages}
\item{Describe the two week anual summer camp and activities related
  to RFI exscision, professional development, and STEM retention}
\item{Tie into INCLUDES program}
\end{itemize}

\subsection{Project Plan and Timeline}
\label{sec:plan}

\begin{itemize}
\item{\textbf{Section lead(s):} Laura Jensen}
\item{\textbf{Target length:} 2 pages}
\end{itemize}

\subsubsection{Work to be Undertaken OR Key Milestones and Evaluation}
\label{sec:milestones}

\begin{itemize}
\item{Should align with activities identified in
    \S\ref{sec:innovation}}
\item{Should also include metrics for success and a plan for evaluation}
\end{itemize}


\subsubsection{Proposed Timeline}
\label{sec:timeline}

\begin{itemize}
\item{A graphical timeline, organized by year and work type}
\item{A narrative description of the timeline}
\end{itemize}

\subsection{Results from Prior NSF Support}
\label{sec:prior_support}

\begin{itemize}
\item{\textbf{Section lead(s):} All (as needed)}
\item{\textbf{Target length:} ? (must be $< 5$ pages)}
\item{\textbf{Note: Only needed for PIs or co-PIs with a current NSF
      award or one with an end date in the past five years.}}
\item{For each award:}
  \begin{itemize}
  \item{NSF Award number, amount, and period of support}
  \item{Title of project}
  \item{Summary of completed/proposed work}
    \begin{itemize}
    \item{Intellectual Merit}
    \item{Broader Impacts}
    \end{itemize}
  \item{List of publications}
  \item{Evidence of research projects and their availability}
  \item{Relation of completed work to proposed work (for renewals
      only)}
  \end{itemize}
\end{itemize}

\end{document}
