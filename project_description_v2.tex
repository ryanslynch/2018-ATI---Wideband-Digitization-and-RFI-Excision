\documentclass[10pt]{myNSF}

\usepackage{amssymb}
\usepackage{mypack}

\begin{document}

\title{An Ultrawideband Digital Signal Processing System for the Green
  Bank Telescope}
\maketitle

\section{Project Description}
\label{sec:project_description}

\begin{itemize}
\item{\textbf{Note: 15-page limit}}
\item{\textbf{Note: A separate, 2-page data management plan can
      include details on standards used for data and metadata, and
      policies for accessing, sharing, and re-using data}}
\item{\textbf{Note: A separate resources and facilities section can
      include a ``description of the internal and external resources
      (both physical and personnel)'', and this may be a good place to
      discuss local expertise.  We must also include biographical
      sketches that list education, professional preparation, and
      related ``products'', somewhat akin to a CV}}
\end{itemize}

\subsection{Overview}
\label{sec:overview}

We propose to use state-of-the-art technologies to build an
ultra-wideband digital signal processing (UWB-DSP) system that will be
integrated into a new radio receiver under development for the Robert
C.\ Byrd Green Bank Telescope (GBT) at the Green Bank Observatory
(GBO).  Our UWB-DSP system will consist of fast, **high** \textbf{(is
  this the right word?)} bit-depth analog-to-digital converters (ADCs)
that will directly sample the entire $0.7$--$4\; \GHz$ radio frequency
(RF) bandwidth of the new ultra-wideband receiver (UWBR), bypassing the
GBT's usual system of analog mixers and filters that convert the RF
signal to an intermediate frequency (IF).  The UWB-DSP system will
also make use of new system-on-chip (SOC) architectures to enable
advanced, real-time radio frequency interference (RFI) excision using
statistical and new machine learning (ML) algorithms to identify RFI
in the presence of astronomical signals.  The combination of real-time
RFI removal and high dynamic range digitization as close as possible
to the front-end receiver will make this complete UWB system
**significantly** \textbf{(better word choice?)} more resistant to RFI
than is currently possible with existing technology on the GBT.  This
is crucially important given the experiences of a similar UWB system
that have been deployed on the Effelsberg Radio Telescope that were
crippled by strong interference.

The primary science motivation for our UWB system is the direct
detection of low-frequency gravitational waves (GWs) via pulsar
timing.  The system will also allow for new, wideband spectral studies
of fast radio bursts (FRBs), magnetars, and other radio transients, as
well as faster surveys of regions rich in molecular lines at these
frequencies (e.g. H{\sc II} radio recombination lines and complex
chemical species).  The UWB-DSP system will use cutting edge digital
hardware, innovative RFI excision techniques not in use at any other
observatories, and pioneering methods for handling very high data
rates using 100-gigabit Ethernet (GbE) protocols.  These will
complement the UWBR, which will deliver a combination of wider
instantaneous bandwidth and lower system noise temperature than was
possible with previous generation technology.  Our project will thus
pair advanced digital and analog technologies for the world's largest
single-dish radio telescope to enable transformative scientific
advances in cutting edge fields of astronomy and astrophysics.

The UWB system will be deployed on the GBT as a facility instrument
open to the full astronomical community via the NSF-funded open skies
program.  This project will also serve as a pilot program for upgrades
to the GBT's existing receivers and IF system.  All of the hardware
design, firmware, and software developed through our efforts will be
made publicly available for use at other observatories, and will be
directly relevant for possible future telescopes such as the Next
Generation Very Large Array (ngVLA) and Square Kilometer Array (SKA).
We will also leverage GBO's leadership in the NSF INCLUDES program to
broaden participation in digital engineering and radio astronomy via
an annual summer camp for undergraduate students.  During this camp
students will directly participate in developing our new RFI excision
project by creating training data sets for machine learning
algorithms.  The students will also be exposed to a wide range of
engineering and scientific disciplines that contribute to the success
of GBO, and will receive interventions that will increase retention in
STEM fields \textbf{(among rural/first generation college students?)}.
This will create a pipeline of students for GBO's successful summer
student programs (including our NSF-funded REU program), alumni of
which have already contributed to the RFI excision project.  The
UWB-DSP project will thus have an extremely broad impact on the wider
scientific community and the next generation of STEM professionals.

\subsection{Intellectual Merit}
\label{sec:IM}

\subsubsection{Motivation}
\label{sec:motivation}

\begin{itemize}
\item{\textbf{Section lead(s):} Ryan Lynch, digital group}
\item{\textbf{Target length:} 3 pages}
\item{\textbf{Note: Highlight ``expected significance'' here}}
\item{\textbf{Note: Highlight ``objectives'' here}}
\end{itemize}

A diverse range of high-impact astronomical science benefits from
ultra-wide bandwidths at radio frequencies, particularly the study of
pulsars and transients, and of rich molecular line regions.  At the
same time, radio astronomy must increasingly find technical solutions
for sharing the radio spectrum with civil and commercial users.  We
describe the science and technical motivations for this project in
more detail below.

\subsubsubsection{Scientific Motivation}
\label{sec:science_motivation}

\alanheading{The Low-Frequency GW Universe} The primary science driver
of the UWB-DSP system is the direct detection of nanohertz frequency
GWs via pulsar timing, which is the focus of the NSF-supported North
American Nanohertz Observatory for Gravitational Waves Physics
Frontier Center (NANOGrav PFC).  At these GW frequencies the dominant
source class is expected to be supermassive binary black holes (SMBBH)
in the early stages of inspiral at the centers of galaxies; exotic
sources of GWs such as cosmic strings may also emit in the
$\nHz$-regime.  The NANOGrav PFC and similar experiments are highly
complementary to ground- and space-based laser interferometers that
probe higher GW frequencies, and cosmic microwave background
experiments that probe lower frequencies.  The NANOGrav PFC is
on-track to detect a stochastic GW background within the next 3--5
years, and is already placing important constraints on the amplitude
of this background that informs models of SMBBH evolution and coupling
to the surrounding galactic environments.  The detection of individual
continuous wave sources is expected to follow in the coming decade,
which will enable multi-messenger studies of SMBBH systems.  The
NANOGrav PFC also places the most stringent existing limits on the
energy density of cosmic strings.

The NANOGrav PFC uses the GBT and the William E. Gordon telescope at
the Arecibo Observatory (AO) to observe a pulsar timing array (PTA) of
millisecond pulsars (MSPs) distributed across the sky.  The extremely
high rotational stability of MSPs allows them to be used as clocks
whose ``ticks'' are pulse times of arrival (TOAs) that can be
accurately measured \emph{and predicted} with accuracies of $\lesssim
100\; \ns$ over time scales of decades.  The influence of GWs at the
Earth will cause a $10$--$100\; \ns$ deviation in the TOAs because of
the changing path-length between the observer and the pulsars.  The
quadripolar nature of GWs will specifically cause a quadripolar
angular correlation between pulsars distributed across the sky, which
makes it possible to distinguish between GWs and other sources of TOA
deviations (e.g. uncorrelated effects on individual pulsars, or
observatory clock errors or Solar System effects that will have a
monopolar and dipolar angular correlation, respectively).  This
process of pulsar timing demands a complete characterization of the
MSPs themselves, including their rotational, astrometric, and binary
properties.  Thus, other high-impact science emerges from this
project, such as pulsar mass measurements, tests of general
relativity, and novel constraints on Solar system planetary
ephemerides.

One of the most important steps in obtaining TOAs with the required
accuracy is measuring and correcting for the effects of the ionized
interstellar medium (ISM) on pulsar signals.  One of these effects is
a dispersive delay given by
\begin{equation}
  \Delta t_{\rm DM} = k_{\rm DM} \, \DM(t,f)\, \left( f_{\rm l}^{-2} -
  f_{\rm h}^{-2} \right) ,
\end{equation}
where $k_{\rm DM}$ is a physical constant, $f$ is the radio frequency
and the subscripts denote a lower and higher frequency, and $\DM(t,f)$
is the dispersion measure, i.e. column density of free electrons
between the pulsar and the Earth.  We emphasize that \DM\ is both time
and frequency dependent (see \textbf{***}), and thus represents a
noise term \emph{that must be measured at each observing epoch with a
  fractional precision of $\sim ***$}.

The NANOGrav PFC currently employees a two-receiver strategy at the
GBT to precisely measure DM, observing from $0.72$--$0.92\; \GHz$ and
$1.1$--$1.9\; \GHz$.  This approach is sub-optimal for several reasons.
First, it effectively doubles the observing time needed to obtain a
single TOA.  Second, for operational reasons these observations are
typically scheduled with a separation of a few days, making it
impossible to resolve DM variations on shorter timescales.  \emph{The
  UWB system will double the observational efficiency of
  high-precision pulsar timing programs while providing a factor of
  \textbf{***} improvement in measurements of DM.  When coupled with
  higher pulsar signal-to-noise from the wider instantaneous
  bandwidth, the NANOGrav PFC's sensitivity to GWs will increase at
  twice the rate as without an UWB system}.  This in turn will
effectively double the volume over which the NANOGrav PFC is sensitive
to individual SMBBHs---analogous to the improvement between the first
phase of the Laser Interferometric Observatory for Gravitational Waves
(LIGO) and Advanced LIGO.

\alanheading{Radio Transients} Wide instantaneous bandwidth is
essential for characterizing the spectro-temporal behavior of highly
variable radio transients.  One such population are fast radio bursts
--- millisecond duration radio-frequency pulses that originate in
distant galaxies.  Their physical origin is one of the most pressing
mysteries in astronomy and will be a major area of research in the
coming decade.  To-date, only one FRB has been observed to repeat (FRB
121102), a fact which has enabled the only precise interferometric
localization of an FRB to a host galaxy, as well as long-duration
study of the changing characteristics of the bursts (e.g. DM, Faraday
rotation measure (RM), and burst morphology).  With telescope like the
Australian SKA Pathfinder and the Canadian HI Intensity Mapping
Experiment poised to discover dozens (if not hundreds) of new FRBs,
more repeaters are sure to follow.

FRB 121102 exhibits dramatic burst-to-burst spectro-temporal variation
including a) a highly variable power-law spectral index;b)
non-power-law spectral shapes including band-limited bursts with
characteristic bandwidths of $\sim***\ \GHz$; c) changing peak
frequency; d) changing burst morphology; and e) distinct sub-bursts
that drift towards lower peak frequencies with time in the larger
burst envelope.  These features may be intrinsic, extrinsic, or
both --- the sub-burst structure in particular may be a sign of plasma
lenses in the local environment of FRB121102.  There is also some
evidence for secular changes in DM and RM.  All bursts thus far have
been detected between $\sim 1$--$8\; \GHz$ despite significant
observing campaigns at lower and higher frequencies.

Any theory regarding the nature of FBR 121102 (and presumably at least
some class of FRBs more generally) must explain these properties, so
they serve as a powerful diagnostic tool for understanding FRBs'
physical origins.  However, most burst detections are limited by the
bandwidth of the receiver, so the only way thus far to investigate the
behavior of FRB 121102 over ultra-wide bandwidths has been through
simultaneous observations using multiple telescopes.  This is
obviously logistically complicated and sub-optimal.  Our new UWB system
will enable spectro-temporal studies of FRB 121102 and future
repeating FRBs, answering critical questions such as a) does the
characteristic bandwidth of bursts change with frequency, and if so,
with what form? b) do band-limited bursts appear simultaneous in
widely separated sub-bands? c) do sub-bursts cluster in frequency and
time, or can the peak frequency change on burst-to-burst timescales?
d) what is the burst morphology over ultra-wide bandwidths? e) is the
apparent $\sim 1\; GHz$ lower limit real or an artifact of
under-sampling at lower frequencies? and f) does DM vary as a function
of frequency in broad-band bursts?  The answers to these questions can
then be quantitatively compared with physical models for FRB emission,
such as the aforementioned plasma-lensing model.

A second class of variables are radio magnetars --- a sub-class of
neutron stars whose emission is powered by the decay of extremely
strong magnetic fields.  To-date only four radio magnetars have been
discovered (out of a larger population of \textbf{***} magnetars that
emit X-rays and gamma-rays).  Their sporadic emission and variable
power-law spectral index, polarization fraction and position angle,
and burst morphology stand in stark contrast to rotation-powered radio
pulsars, and the physical processes giving rise to their radio
emission remains a mystery.  As with FRB 121102, spectro-temporal
studies have been limited by the relatively small bandwidth provided
by most receivers.  Interestingly, there may be a connection between
magnetars and FRBs.  A young, powerful magnetar is one of the leading
candidates for the source of FRB 121102, and the extremely high RM
observed in FRB 121102 has only one known analog: the radio magnetar
near the center of the Milky Way.  Thus, studies of magnetars may
improve our understanding of FRBs, and vice versa.  The UWB-DSP system
will thus be a powerful tool for expanding our knowledge of radio
transients.

\alanheading{Molecular Line Surveys} \textbf{TODO}

\subsubsubsection{Technical Motivation}
\label{sec:technical_motivation}

The ultra-wide bandwidth observations needed to realize the above
scientific potential come with a number of technical challenges.  One
of the most pressing is the ability to \emph{share the spectrum} with
man-made transmitters, producing scientifically usable data at
frequencies where there is significant, strong RFI.  GBO's location at
the center of the 13,0000 square-mile National Radio Quiet Zone and
smaller West Virginia Radio Astronomy Zone gives it unique
interference protection, but many source of RFI, such as satellite
transmitters, are unavoidable (for more information on these
interference protection zones see Facilities, Equipment, and Other
Resources).  We classify these capabilities into RFI resistance (i.e.,
a high linear dynamic range in every component of the front-end and
back-end) and RFI excision (i.e. removal of RFI at the lowest-possible
level of data to improve data quality).  We note that GBO is taking
extreme care to ensure that the front-end UWBR is sufficiently
resistant to RFI as part of a separate research and development
effort, so here we concern ourselves only with the UWB-DSP system.

\alanheading{RFI Resistance} The GBT currently uses the Versatile
Green Bank Astronomical Spectrometer (VEGAS, developed in part with
support from \textbf{NSF-***}) as its primary digital back-end system.
VEGAS uses eight spectrometer banks each consisting of $2 \times 3\;
\mathrm{Gsps}$ 8-bit ADCs (one for each polarization channel) paired
with a high-performance computer (HPC) equipped with an nVidia GTX 780
graphical processing unit (GPU).  A relatively straightforward
expansion of the HPC system will be sufficient to process the full
bandwidth provided by the UWBR for pulsar and FRB observations, but
this approach comes with significant drawbacks.  Most notably, VEGAS
makes use of the GBTs IF system before digital sampling, which will
expose the UWB system to potential saturation of numerous components
including the RF-over-fiber transceivers, two additional frequency
mixers and bandpass filters, and the VEGAS 8-bit ADCs.  In recent
months GBO staff have become aware of total power instabilities that
are present at a variety of observing bands (most notably L-Band,
i.e. $1$--$2\; GHz$) and digital back-end systems in addition to VEGAS.
These instabilities correlate with the presence of strong RFI around
$1.09\; \GHz$ and have been isolated to analog components between the
RF-over-fiber transceivers and second frequency mixer.  The UWB system
will be exposed to an even worse RFI environment that includes
cellular communication towers, digital television transmitters,
Global Positioning System satellites, airport radar and aircraft
positioning systems, Iridium communication satellites, and Sirius XM
Satellite Radio.  \emph{We thus have empirical evidence that
  illustrates the need to minimize the analog components in the UWB
  signal path and to digitally sample with sufficient dynamic range
  to maintain linearity across $0.7$--$4\; \GHz$}.

The UWB-DSP system will accomplish this goal by completely bypassing
the existing GBT IF system, sampling at RF with a minimum of 12-bits
per polarization channel.  This will also provide better spectral
baseline stability.

\alanheading{RFI Excision} GBO has been actively testing several
techniques for automated RFI detection and excision.  These using
median absolute deviation of complex voltage samples, spectral
kurtosis, and a new project using machine learning (ML) algorithms.
The latter is, to our knowledge, unique among major radio astronomy
observatories.  As part of normal operations the GBT regularly
conducts RFI scans that have resulted in a database of over 20 million
instances of narrow-band RFI, resulting in a rich data set that can be
used to train deep neural networks.  However, the limitations of our
existing ROACH2-based DSP hardware prevents us from fully
implementing, testing, and deploying these RFI excision techniques.
\textbf{(For Luke: can you provide more detail?  What are the limit
  factors?)}

Luckily, new developments in field programmable gate arrays (FPGAs) and
associated tool flows fundamentally change the paradigm for realizing
sophisticated RFI excision techniques.  \textbf{(For Luke: More detail
  would also be good here.  Specify the new hardware and development
  capabilities and what they enable.)}

\subsubsection{Innovation}
\label{sec:innovation}

\begin{itemize}
\item{\textbf{Section lead(s):} Digital group (+ BTL/CASPER?)}
\item{\textbf{Target length:} 4 pages}
\item{\textbf{Note: Highlight ``relationship to present state of
      knowledge'' here}}
\item{\textbf{Note: To the extent possible, explicitly describe work
      to be undertaken and/or major activities (can expand in
      \S\ref{sec:plan} as needed)}}
\item{\textbf{Note: We could include impact on key science drivers
      here}}
\end{itemize}

To realize the above scientific and technical goals we will make use
of cutting-edge hardware and innovative DSP techniques.  \emph{We
  emphasize that we will use modular designs that abstract away
  lower-level components.  This will make it easier to rapidly take
  advantage of new technologies as they emerge, so that the impact of
  our efforts will last much longer than a single generation of
  specific technology.}

\subsubsubsection{Hardware}

\alanheading{Analog-to-Digital Converters}
\begin{itemize}
\item{New commercial options}
\item{Challenges posed by interleaving}
  \begin{itemize}
  \item{Highly work done on ADC calibration as part of VEGAS}
  \end{itemize}
\end{itemize}

\alanheading{System-on-Chip}
\begin{itemize}
\item{New commercial options}
\item{Provide some rough metrics for improvement over ROACH2}
  \begin{itemize}
  \item{Logic density}
  \item{Memory and memory bandwidth}
  \item{Speed}
  \item{Power consumption}
  \item{Highly what the new advances all of the above enable}
  \end{itemize}
\item{Custom integration of ADCs and SoC boards}
\end{itemize}

\subsubsubsection{Firmware Development}

\begin{itemize}
\item{Is the typical DSP stuff (e.g. PFBs) all straight forward?)}
\end{itemize}

\alanheading{High-bit Sampling}

\begin{itemize}
\item{Do we need to discuss how to deal with bandpass slope and
    selecting appropriate bits?}
\end{itemize}

\alanheading{Custom Encoding Protocols}

\begin{itemize}
\item{100 GbE, duplex 100 GbE, DWDM 100 GbE, PCIe $3 \times 8$, PCIe
    $3 \times 16$, PCIe $4 \times 8$, FMC, FMC+}
\end{itemize}

\alanheading{Interfacing with Hardware and Peripherals}

\begin{itemize}
\item{Custom and/or COTS hardware (ADCs, GbE) FMC cards with FPGA boards}
\item{Programmable software to programmable logic}
\item{FPGA with IO protocols}
\item{RFI Excision Blocks}
\item{DACs to CASPER tool flow}
\end{itemize}

\subsubsubsection{Protocols and Formats for High Data Rates}

\begin{itemize}
  \item{\textbf{Question for digital group: How would we break up band
        (i.e. sub-band the way GUPPI and VEGAS do in coherent DD modes)
        and transmit to HPCs?}}
  \item{\textbf{Note: We should talk to Chris and computing group
        about network infrastructure}}
\item{Suitability of duplexed or DWDM 100 GbE links}
\item{Network topology based on ``few'' 100 GbE links as opposed to
    ``many'' 10 GbE links}
\item{Reliable, fast, low-latency generalized packet formats for
    relaying high-speed, high bit-depth ADC samples from receiver to
    DSP system}
\item{Evaluation of a PCIe-mounted FPGA-based card suitable for larger
    systems}
\end{itemize}


\subsubsubsection{Active RFI excision}

\alanheading{Impulsive RFI Mitigation} GBO has built upon work started
at the Nan\"{c}ay Radio Observatory that detects and excises
interference from ground-based RADAR sources.  \textbf{(For Luke:
  Please provide a few sentences describing the technique and the
  testing that has been done so far.)}  

\alanheading{Spectral Kurtosis} Initially conceived at the Center for
Solar-Terrestrial Research at New Jersey Institute of Technology as a
robust statistical RFI detector, the simple sum/sum-squared algorithm
lends itself naturally to implementations in FPGAs. \textbf{(For Luke:
  Need a general description of the technique.)} Over the past year, a
collaboration between GBO and West Virginia University has created a
Python-based implementation of the generalized spectral kurtosis
estimator (source 1). However, our current implementation is not
real-time, and has been designed specifically on archived (and
extensively analyzed) pulsar data.  While its overall effectiveness
has been proven, more evaluation is required and we are limited in our
ability to do so.  Our current data is too small, too old, and too
disparately sampled to allow our tests to arrive at a high level of
certainty of its effectiveness under varying conditions (especially
when initialization/calibration \textbf{(For Luke: What initialization
  and calibration?)} is taken into account). Additionally, none of our
existing back-end systems have enough overhead \textbf{(For Luke: What
  does overhead refer to here?)} in the FPGAs to allow an HDL
(\textbf{For Luke: Define HDL.)}  implementation to co-exist with the
existing firmware.  New, larger hardware is needed test and deploy a
real-time implementation of this method that could then be shared with
the wider community.

\alanheading{Machine Learning} ML is subset of artificial intelligence
whose applicability and accessibility has increased dramatically in
recent years. In response to industry trends, Xilinx has recently
taken steps to optimize their hardware (see Zynq Ultrascale+, Alveo,
and Versal hardware lines) and software (see Deephi acquisition,
Xilinx ML Suite, SDAccel) for easier development and faster run-time
of ML algorithms.  While ML is being used to detect and characterize
astronomical sources, there have been few attempts at what could be a
relatively ideal and straightforward application of the technology to
existing problems of real-time RFI detection, classification, and
mitigation.  In addition, as a by-product of an ongoing decadal
analysis of RFI at the GBO site, we have access to 40 million samples
of sortable and searchable RFI instances, which will serve as a
starting point for a supervised learning implementation.

\textbf{(For Luke: I think we need to discuss the specific techniques
  we might use in more detail)}

\alanheading{Verification and Qualification} While being able to
accurately and precisely detect/remove RFI is an important and
difficult problem to solve, it is not necessarily more difficult or
important than defining a generic test procedure or methodology for
ensuring/testing the general efficacy of specific removal techniques -
or their ability to preserve the underlying scientific data of
interest. \textbf{For Luke: What do you propose specifically to do
  here?)}

\subsubsubsection{Interference compliance}
\begin{itemize}
\item{Design of low-power, non-interfering electronics}
\item{Shielding (w/ input from Carla?)}
\end{itemize}

\subsubsubsection{Power Supply and Cooling} \textbf{Not sure if this
  section is needed}

\subsubsubsection{Multi-mode Observing} \textbf{Not sure if this
  section is needed}
\begin{itemize}
\item{\textbf{Note: We should talk with software group about
      software back-ends}}
\end{itemize}

\subsection{Broader Impacts}
\label{sec:BI}

\textbf{Note: We should decide if we want to support a postdoc,
  intern, or grad student.  This would most likely be in engineering.
  If so, we can place this under sections for ``Development of a
  Competitive Workforce''}

\subsubsection{Commitment to the Public}
\label{sec:commitment}

The UWB-DSP system will be deployed on the GBT as a
facility-supported, general purpose instrument available to all GBT
users.  A majority of GBT time is allocated through the NSF-funded
open-skies program, and is thus open to astronomers anywhere in the
world.  The other primary users of the GBT are the Breakthrough Listen
project and the NANOGrav PFC.  The importance of the UWB system to the
NANOGrav PFC has been explained in \S\ref{sec:science_motivation}, and
it will also be valuable to Breakthrough Listen, as it will allow for
faster surveys for extraterrestrial techno-signatures.  Both NANOGrav
and Breakthrough Listen have committed to making data publicly
available. (\textbf{(How much should we go into detail on this?)}

All of the hardware designs, firmware, and software produced in the
course of this work will be made freely available to the wider
astronomical and radio science communities for use at other
facilities.  We will use a mix of technical memos, presentations at
conferences, and refereed publications to document and communicate the
results of the work to the broadest possible audiences.  GBO also
hosts thousands of visitors each year through its education and
outreach programs.  Visitors participate in public tours (some of
which are specialized for a technical audience), short educational
courses, and weekend and week-long student camps.  The
co-investigators all participate in these programs and will use these
opportunities to educate the broader public about the UWB system and
radio astronomy more generally.

\subsubsection{Enhanced Infrastructure for Research and Education}
\label{sec:infrastructure}

\begin{itemize}
\item{\textbf{Section lead(s):} Ryan Lynch + scientific staff, digital
    group}
\item{\textbf{Target length:} 2 pages}
\end{itemize}

\subsubsubsection{Maximizing Return from the UWB Receiver}

The UWB-DSP system will be fully integrated into the UWB front-end
receiver.  This will allow us to bypass the existing GBT IF system,
mitigating the risk of non-linear response in the analog components
caused by strong RFI.  By digitizing as close as possible to the
receiver we will also minimize gain fluctuations that can lead to
unstable spectral baselines.  These benefits taken together with the
active RFI identification and excision algorithms will ensure that the
UWB system results in the highest quality science data products under
all observing conditions.

\subsubsubsection{A Pilot Program for Future GBT Upgrades}

The GBT has a flexible IF system that has enabled ground-breaking
discoveries in all areas of astronomy, but it is now over 20 years old
and has several limitations.
\begin{itemize}
  \item{Receivers operating above $12\; \GHz$ could provide $>8\; GHz$
    of instantaneous bandwidth but are limited by various bandpass
    filters to no more than $8\; \GHz$ bandwidth, and in many cases
    only $4$--$6\; \GHz$ (see Table \ref{table:rx_bandwidth} for
    details).}
  \item{Multiple spectral windows (up to 64) are formed via a complex
    set of secondary and tertiary mixers and bandpass filters before
    the signal is finally sampled at base-band frequencies.  Once
    again, bandpass filters are a limiting factor, setting a maximum
    separation between spectral windows.  The number of converters
    also limits the maximum number of spectral windows.}
  \item{The secondary and tertiary converters are housed in a building
    over 1-km from the GBT.  The signals are transported via
    RF-over-fiber links which are subject to saturation.}
  \item{All of the above analog components can undergo gain variation
    due to changing environmental conditions.  For very deep
    observations of faint sources the resulting spectral baseline
    changes can limiting.}
  \item{Doppler broadening of spectral lines caused by the Earth's
    motion can be removed by a tunable first-stage frequency mixer,
    but only for a single rest frequency.  More complex Doppler
    tracking (e.g. to account for source motion in a binary system) is
    not possible.}
\end{itemize}

The digital technology that we propose to use in the UWB-DSP system
has the potential to eliminate nearly all of these restrictions.
Multiple fast ADCs could be employed to sample the full available
bandwidth for single-pixel receivers, and would provide maximum
flexibility when trading bandwidth for pixels in multi-pixel
receivers.  \emph{This could lead to as much as a factor of 20
  increase in survey speed when observing widely spaced spectral lines
  (see Table \ref{table:rx_bandwidth})}.  Digitization would occur
either at RF for low-frequency receivers and either in higher Nyquist
zones or after a single down-conversion at higher frequencies.  This
would eliminate most analog components, drastically lowering the risk
of saturation from RFI while providing much better spectral baselines.
Doppler tracking and windowing would be accomplished digitally.
Active RFI mitigation would also be incorporated into all GBT
observing.  This will become increasingly important as more of the
super high and extremely high frequency portions of the spectrum are
used for new technologies (e.g. collision avoidance car RADARs that
operate in the $76$--$81\; \GHz$ band).

\textbf{(TODO: A paragraph on scientific impact)}

\emph{This would be a transformational modernization of
  the GBT, analogous to the upgrades of the ``extended'' Jansky Very
  Large Array, and would revolutionize all areas of GBT science.}  The
UWB-DSP system is a pathfinder that will allow us to determine the
most effective and affordable solutions for these future upgrades.

\subsubsubsection{Relevance for Other Facilities}

The radio astronomy community is planning for major new facilities
that will begin construction and operation in the 2020s, such as the
SKA and ngVLA, in addition to a myriad of experiment-class
instruments.  These next generation telescopes should use next
generation technology, including integrated wideband digitization and
RFI excision.  We emphasize once again that all products supported by
this ATI proposal will be made freely available to the wider
community, including communication protocols, firmware designs, RFI
excision implementations, and tool flows.  Because of our focus on
modular design enabling rapid development, our efforts will also serve
as a pathfinder for technologies that will be deployed on these future
facilities.  \emph{The GBT, and single-dish telescopes more broadly,
  are perfect test-beds for new techniques and technologies because of
  their simpler design and signal paths (compared to multi-dish
  arrays).}

\subsubsection{Broadening Participation}
\label{sec:participation}

\begin{itemize}
\item{\textbf{Section lead(s):} Sue Ann Heatherly}
\item{\textbf{Target length:} 2 pages}
\item{Describe the two week annual summer camp and activities related
  to RFI excision, professional development, and STEM retention}
\item{Tie into INCLUDES program}
\end{itemize}

\subsection{Project Management Plan}
\label{sec:plan}

\begin{itemize}
\item{\textbf{Section lead(s):} Laura Jensen}
\item{\textbf{Target length:} 2 pages}
\item{Should align with activities identified in
    \S\ref{sec:innovation}}
\item{Should also include metrics for success and a plan for evaluation}
\item{Can we talk about modular design and future upgrade strategies
  here?}
\end{itemize}

\subsubsection{Work to Be Undertaken}
\label{sec:milestones}

\subsubsection{Evaluation}
\label{sec:evaluation}

\subsubsection{Timeline}
\label{sec:timeline}

\begin{itemize}
\item{A graphical timeline, organized by year and work type}
\item{A narrative description of the timeline}
\end{itemize}

\subsection{Results from Prior NSF Support}
\label{sec:prior_support}

\begin{itemize}
\item{\textbf{Section lead(s):} All (as needed)}
\item{\textbf{Target length:} ? (must be $< 5$ pages)}
\item{\textbf{Note: Only needed for PIs or co-PIs with a current NSF
      award or one with an end date in the past five years.}}
\item{For each award:}
  \begin{itemize}
  \item{NSF Award number, amount, and period of support}
  \item{Title of project}
  \item{Summary of completed/proposed work}
    \begin{itemize}
    \item{Intellectual Merit}
    \item{Broader Impacts}
    \end{itemize}
  \item{List of publications}
  \item{Evidence of research projects and their availability}
  \item{Relation of completed work to proposed work (for renewals
      only)}
  \end{itemize}
\end{itemize}

\end{document}
